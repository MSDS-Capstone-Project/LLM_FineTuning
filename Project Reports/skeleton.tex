\documentclass[11pt,twocolumn]{article}
\usepackage[margin=1in]{geometry}
\usepackage{graphicx}
\usepackage{amsmath,amssymb}
\usepackage{hyperref}
\usepackage{natbib}
\newcommand{\coursefootnote}{
\renewcommand{\thefootnote}{\fnsymbol{footnote}}
\footnotetext[1]{Course Project for DS-6051 -- Decoding Large Language Models.}
\renewcommand{\thefootnote}{\arabic{footnote}}
}
\title{
\textbf{TITLE OF YOUR REPORT}
}
\author{
Student Name \\
University/Department \\
\texttt{email@domain.edu}
}
\date{}
\begin{document}
\maketitle
\coursefootnote % This inserts the footnote exactly once
\begin{abstract}
A concise summary (150--300 words) of the problem, methods, and key results.
\end{abstract}
\section{Introduction}
\label{sec:intro}
Provide the context, motivation, and main objectives.
\section{Related works}
\label{sec:background}
Summarize background. Highlight key papers and how they relate to your work.
\section{Methodology}
\label{sec:methodology}
Explain data, model(s), approach, and experimental setup.
\section{Experiments and Results}
\label{sec:experiments}
Present quantitative and qualitative findings. Use figures/tables if needed.
\section{Discussion}
\label{sec:discussion}
Interpret results, highlight limitations, analyze interesting points.
\section{Conclusion and Future Work}
\label{sec:conclusion}
Summarize contributions and propose next steps.
\bibliographystyle{plainnat}
\begin{thebibliography}{99}
\bibitem{some-key-2023}
Author, A. (2023).
\textit{Title of the reference or paper.}
Journal or Conference Name, pages, or URL.
\end{thebibliography}
\appendix
\section{Appendix (Optional)}
Extra details, code snippets, extended tables or figures.
\end{document}\bibliographystyle{plainnat}
\begin{thebibliography}{99}
\bibitem{some-key-2023}
Author, A. (2023).
\textit{Title of the reference or paper.}
Journal or Conference Name, pages, or URL.
\end{thebibliography}
\appendix
\section{Appendix (Optional)}
Extra details, code snippets, extended tables or figures.